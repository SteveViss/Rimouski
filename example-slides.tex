\documentclass{eecslides}

\usepackage[frenchb]{babel}

\title[Modèle \emph{beamer}]{Modèle de présentation \emph{beamer}}
\subtitle{Le style ``Rimouski''}
\author[Tim \& Steve]{Timothée Poisot \and Steve Vissault}
\institute[Chaire de recherche EEC]{UQAR -- \emph{Theoretical Ecosystem Ecology}}
\date{\today} % date

\begin{document}

	\begin{frame}[plain]
		\titlepage
	\end{frame}

	\section{Introduction}

	\begin{frame}{Titre de la diapositive}
	    
		\begin{itemize}
			\item Énumération
			\item Texte \alert{alerté} 
		\end{itemize}
	
	\end{frame}

	\begin{frame}{Titre de la diapositive}{Sous-titre}
		Une diapo avec un sous-titre

		\begin{enumerate}
			\item Premier élément
			\item Deuxième élément
		\end{enumerate}

		$$\frac{dN_i}{dt} = r_iN_i\sum_j\alert{\alpha}_{ji}N_j$$

	\end{frame}

	\section{Méthodes}

	\begin{frame}[plain]
		Cette frame est \emph{vide}, en utilisant l'argument optionnel \texttt{plain}.
	\end{frame}

	\begin{frame}[allowframebreaks]{Frame multiple}
		Cette frame est coupée en plusieurs parties.

		\framebreak

		Par exemple, ça, c'est la deuxième partie.
	\end{frame}

	\section{Discussion}

	\begin{frame}[t]{Frame centrée en haut}
	    Cette frame est alignée par le haut. Ça peut servir, sans doute.	
	\end{frame}

\end{document}