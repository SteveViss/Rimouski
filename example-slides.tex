\documentclass{eecslides}

\usepackage[frenchb]{babel}

\title[Modèle \emph{beamer}]{Modèle de présentation \emph{beamer}}
\subtitle{Le style ``Rimouski''}
\author[Tim \& Steve]{Timothée Poisot \and Steve Vissault}
\institute[Chaire de recherche EEC]{UQAR -- \emph{Theoretical Ecosystem Ecology}}
\date{\today} % date

\begin{document}

	\begin{frame}[plain]
		\titlepage
	\end{frame}

	\section{Introduction}

	\begin{frame}{Titre de la diapositive}
	    
		\begin{itemize}
			\item Énumération
			\item Texte \alert{alerté} 
		\end{itemize}
	
	\end{frame}

	\begin{frame}{Une diapo avec des maths}{Sous-titre}
		Une diapo avec un sous-titre.

		\begin{enumerate}
			\item Premier élément
			\item Deuxième élément
		\end{enumerate}

		$$\frac{dN_i}{dt} = r_iN_i\sum_j\alert{\alpha}_{ji}N_j$$

	\end{frame}

	\section{Méthodes}

	\begin{frame}[plain]
		Cette frame est \emph{vide}, en utilisant l'argument optionnel \texttt{plain}.
	\end{frame}

	\begin{frame}[allowframebreaks]{Frame multiple}
		Cette frame est coupée en plusieurs parties.

		\framebreak

		Par exemple, ça, c'est la deuxième partie.
	\end{frame}

	\begin{frame}[t]{Frame centrée en haut}
	    Cette frame est alignée par le haut. Ça peut servir, sans doute.	
	\end{frame}

	\section{Graphiques}

	\begin{frame}{Avec un graphique pgfplots}{On utilise le cycle \emph{Set3}}
		\begin{center}
			\input{plot.tikz}
		\end{center}	
	\end{frame}

	\begin{frame}{Avec une surface}{On utilise la \emph{colormap} \emph{RdGy}}
		\begin{center}
			\begin{tikzpicture}
	\begin{axis}[
		view/h=40,colormap name=sSpectral
		]
		\addplot3[
			surf,
			shader=flat,
			samples=30,
			domain=-3:3,
			y domain=-2:2] {sin(deg(x+y^2))}; 
	\end{axis}
\end{tikzpicture}
		\end{center}	
	\end{frame}

	\begin{frame}[allowframebreaks]{Graphiques}{Le package \texttt{pgfcb}}
	    
		En plus des fichiers pour modifier le style des slides, le \emph{package} \texttt{pgfcb} met à disposition plusieurs palettes de couleur pour les graphiques. Ces palettes sont celles de \emph{ColorBrewer} 2, visibles à \url{http://colorbrewer2.org/}.

		\framebreak

		Chaque palette exise en trois versions: pour les \emph{dotplots}, les histogrammes et autres versions remplies, et les surfaces. Le nom de la palette est préfixé par \texttt{p}, \texttt{h}, et \texttt{s}.

		\framebreak

		Les palettes disponibles sont les suivantes:

		\framebreak

		Chaque \alert{couleur} peut être appelée indépendamment, en utilisant son numéro dans la palette. Par exemple, cette phrase {\color{S34} utilise la couleur \texttt{S34}}, soit la quatrième valeur de la palette \texttt{S3}. 


	\end{frame}

\end{document}