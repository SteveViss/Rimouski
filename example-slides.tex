\documentclass{eecslides}

\usepackage[frenchb]{babel}
\usepackage{lipsum}

\title[Modèle \emph{beamer}]{Modèle de présentation \emph{beamer}}
\subtitle{Le style ``Rimouski''}
\author[Tim \& Steve]{Timothée Poisot \and Steve Vissault}
\website{http://timotheepoisot.fr/}
\institute[Chaire de recherche EEC]{UQAR -- Chaire de Recherche EEC}
\date{\today}

\begin{document}

	\begin{frame}[plain]
		\titlepage
	\end{frame}

	\section{Introduction}

	\begin{frame}{Titre de la diapositive}
	    
		\begin{itemize}
			\item Énumération
			\item Texte \alert{alerté} 
		\end{itemize}
	
	\end{frame}

	\begin{frame}{Une diapo avec des maths}{Sous-titre}
		Une diapo avec un sous-titre.

		\begin{enumerate}
			\item Premier élément
			\item Deuxième élément
		\end{enumerate}

		$$\frac{dN_i}{dt} = r_iN_i\sum_j\alert{\alpha}_{ji}N_j$$

	\end{frame}

	\section{Structure}

	\begin{frame}[plain]
		Cette frame est \emph{vide}, en utilisant l'argument optionnel \texttt{plain}.
	\end{frame}

	\begin{frame}[allowframebreaks]{Frame multiple}
		Cette frame est coupée en plusieurs parties.

		\framebreak

		Par exemple, ça, c'est la deuxième partie.
	\end{frame}

	\begin{frame}[t]{Frame centrée en haut}
	    Cette frame est alignée par le haut. Ça peut servir, sans doute.	
	\end{frame}

	\section{Mise en forme}

	\begin{frame}{Mise en valeur du contenu}{On utilise l'environnement \emph{Block}}
		\begin{center}
			\begin{block}{\emph{Block} standard}
				\small Eiusdem multi ex provinciae cuius apud est vel regale apud atrocium congregati apparatum cuius usibus.
			\end{block}
			\begin{alertblock}{\emph{Block} alerte}
				\small Eiusdem multi ex provinciae cuius apud est vel regale apud atrocium congregati apparatum cuius usibus.
			\end{alertblock}
			\begin{exampleblock}{\emph{Block} exemple}
				\small Eiusdem multi ex provinciae cuius apud est vel regale apud atrocium congregati apparatum cuius usibus.
			\end{exampleblock}
		\end{center}	
	\end{frame}	

	\section{Graphiques}

	\begin{frame}{Avec un graphique pgfplots}{On utilise le cycle \emph{Set3}}
		\begin{center}
			\begin{tikzpicture}
	\begin{loglogaxis}[
		xlabel={Degrees of freedom}, 
		ylabel={$L_2$ Error},
		cycle list name = pS3]
		\addplot coordinates { (5,8.312e-02) (17,2.547e-02) (49,7.407e-03) (129,2.102e-03) (321,5.874e-04) (769,1.623e-04) (1793,4.442e-05) (4097,1.207e-05) (9217,3.261e-06) };
		\addplot coordinates{ (7,8.472e-02) (31,3.044e-02) (111,1.022e-02) (351,3.303e-03) (1023,1.039e-03) (2815,3.196e-04) (7423,9.658e-05) (18943,2.873e-05) (47103,8.437e-06) };
		\addplot coordinates{ (9,7.881e-02) (49,3.243e-02) (209,1.232e-02) (769,4.454e-03) (2561,1.551e-03) (7937,5.236e-04) (23297,1.723e-04) (65537,5.545e-05) (178177,1.751e-05) };
		\addplot coordinates{ (11,6.887e-02) (71,3.177e-02) (351,1.341e-02) (1471,5.334e-03) (5503,2.027e-03) (18943,7.415e-04) (61183,2.628e-04) (187903,9.063e-05) (553983,3.053e-05) };
		\addplot coordinates{ (13,5.755e-02) (97,2.925e-02) (545,1.351e-02) (2561,5.842e-03) (10625,2.397e-03) (40193,9.414e-04) (141569,3.564e-04) (471041,1.308e-04) (1496065,4.670e-05) };
		\legend{$d=2$,$d=3$,$d=4$,$d=5$,$d=6$};
	\end{loglogaxis} 
\end{tikzpicture}
		\end{center}	
	\end{frame}

	\begin{frame}{Avec une surface}{On utilise la \emph{colormap} \emph{Spectral}}
		\begin{center}
			\begin{tikzpicture}
	\begin{axis}[
		view/h=40,colormap name=sSpectral
		]
		\addplot3[
			surf,
			shader=flat,
			samples=30,
			domain=-3:3,
			y domain=-2:2] {sin(deg(x+y^2))}; 
	\end{axis}
\end{tikzpicture}
		\end{center}	
	\end{frame}

	\begin{frame}{Avec des barres}{On utilise la palette \emph{Set3}}
		\begin{center}
			\begin{tikzpicture}
	\begin{axis}[ybar stacked, cycle list name = hS3]
		\addplot coordinates {(0,1) (1,1) (2,3) (3,2) (4,1.5)}; 
		\addplot coordinates {(0,1) (1,1) (2,3) (3,2) (4,1.5)}; 
		\addplot coordinates {(0,1) (1,1) (2,3) (3,2) (4,1.5)};
	\end{axis}
\end{tikzpicture}
		\end{center}	
	\end{frame}

	\begin{frame}[allowframebreaks]{Graphiques}{Le package \texttt{pgfcb}}
	    
		En plus des fichiers pour modifier le style des slides, le \emph{package} \texttt{pgfcb} met à disposition plusieurs palettes de couleur pour les graphiques. Ces palettes sont celles de \emph{ColorBrewer} 2, visibles à \url{http://colorbrewer2.org/}.

		\framebreak

		Chaque palette exise en trois versions: pour les \emph{dotplots}, les histogrammes et autres versions remplies, et les surfaces. Le nom de la palette est préfixé par \texttt{p}, \texttt{h}, et \texttt{s}.

		\framebreak

		Les palettes disponibles sont les suivantes:

		\framebreak

		Chaque \alert{couleur} peut être appelée indépendamment, en utilisant son numéro dans la palette. Par exemple, cette phrase {\color{S34} utilise la couleur \texttt{S34}}, soit la quatrième valeur de la palette \texttt{S3}. 

	\end{frame}



\end{document}